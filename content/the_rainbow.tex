\danceName{The Rainbow}{longDMin}%danceName must be followed by blank line for proper command parsing

{\large Longways double minor; $AABB$ adlib\hfill \Source{Thompson1774}}

\begin{music}
    \smallmusicsize%
    \parindent0mm%
    \raggedbottom%
    \stafftopmarg=1\Interligne%
    \staffbotmarg=1\Interligne%
    \nobarnumbers%
    \generalmeter{\allabreve}%
    \generalsignature{1}%
    \startpiece%
%%%%%%%% PART A %%%%%%%%
    \zleftrepeat%
    \notes\ql9\Dqbl64\Ibl0592\qb05\nbbl0\slur89u1\qb08\tbl0\qb09\Dqbl{10}5\en\xbar%
    \notes\Dqbl42\qu2\Ibu0363\qb03\qb01\qb0d\tbu0\qb06\en\xbar%
    \notes\ql9\Dqbl64\Ibl0592\qb05\nbbl0\slur89u1\qb08\tbl0\qb09\Dqbl{10}5\en\alaligne%
    %
    \notes\Ibl0413\qb04\qb07\qb06\tbl0\qb01\qu2\qu{N}\en\leftrightrepeat%
%%%%%%%% PART B %%%%%%%%
    \notes\ql6\Dqbl78\qlp9\cl{11}\en\xbar%
    \notes\Ibl0{10}63\slur{10}8u1\qb0{10}\qb08\uppz8\qb07\tbl0\uppz6\qb06\Ibl0533\slur57u1\sh5\qb05\qb07\uppz5\qb05\tbl0\uppz3\qb03\en\alaligne%
    %
    \notes\Ibl0{10}33\slur{10}8u1\qb0{10}\qb08\uppz6\qb06\tbl0\uppz3\qb03\Ibl0933\slur99u1\qb09\qb07\sh5\uppz5\qb05\tbl0\uppz3\qb03\en\xbar%
    \notes\Ibl0473\qb04\qb06\sh5\qb05\tbl0\qb07\ql6\qu{d}\en\xbar%
    \notes\Ibl0643\qb06\qb07\qb06\tbl0\qb04\Dqbl58\ql{10}\en\alaligne%
    %
    \notes\Ibl0533\qb05\qb06\qb05\tbl0\qb03\Dqbl46\ql9\en\xbar%
    \notes\Ibl0773\qb07\qb09\qb05\tbl0\qb07\Ibl0673\qb06\qb09\qb03\tbl0\qb07\en\xbar%
    \notes\Ibl0633\qb06\qb05\qb04\tbl0\qb03\qu2\qu{N}\en\leftrightrepeat%
    \endpiece%
\end{music}

Set \& change {\longs}ides \Thaaa back again \Thbaa hands {\longs}ix half round \& back again \Thaab lead down one Cu. Right \& Left at top \Thbab

\HRule
\HRule
\vspace{1em}
\begin{description}
    \item[A1]
    \item[1-2]Popis v českém jazyce píšeme na doby odpovídající hudbě, kterou máme k dispozici
    \item[3-4]Používáme ustálená spojení u typických figur, kde to dává smysl
    \item[1-4]Jinak používáme natolik přesný popis, abychom ho sami pochopili kdybychom tanec neznali
    \item[5-8]Set and turn vlevo
    \item[9-16]Ditto, set and turn vpravo
    \item[B1]
    \item[1-2] Zapisujeme tanec v pořadí, v jakém má být tančen, nepřeskakujeme části
    \item[3-4] 1.P přejde k 2.D a podá levou ruku
    \item[5-6] 1.P přejde k 3.D a podá pravou, pak levou ruku
    \item[7-8] 1.P políbí 3.D na pravou a levou tvář
    \item[9-10] 1.P a 3.D zátočka za obě o celé, 2.P a 3.P step up
    \item[11-20] dtto vše, ale provádí 1.D, na konci 2.D a 3.D step up
    \item[A2]
    \item[1-16] Sides R, set and turn, sides L, set and turn
    \item[B2] = B1
    \item[A3]
    \item[1-16] Arms R, set and turn, arms L, set and turn
    \item[B3] = B1
\end{description}
\todo{Popis tance}