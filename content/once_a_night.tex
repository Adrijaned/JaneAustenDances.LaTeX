\danceName{Once a Night}{longTMin}{thompson1774}%danceName must be followed by blank line for proper command parsing

{\large Longways triple minor; $AABB$ adlib\hfill \Source{Thompson1774}}

\midifyable
\begin{music}
    \smallmusicsize%
    \parindent0mm%
    \raggedbottom%
    \stafftopmarg=1\Interligne%
    \staffbotmarg=1\Interligne%
    \nobarnumbers%
    \generalmeter{\meterfrac68}%
    \generalsignature{-1}%
    \startpiece%
%%%%%%%% PART A %%%%%%%%
    \zleftrepeat%
    \notes\ql8\cl9\Tqbl{10}85\en\xbar%
    \notes\Tqbl358\Tqbl468\en\xbar%
    \notes\Tqbl5{11}{10}\Tqbl987\en\xbar%
    \notes\Tqbl854\Tqbu321\en\xbar%
    \notes\ql8\cl9\Tqbl{10}85\en\xbar%
    \notes\Tqbl358\Tqbl468\en\alaligne%
    %
    \notes\Tqbl5{11}{10}\Tqbl987\en\xbar%
    \notes\qlp8\qup1\en\leftrightrepeat%
%%%%%%%% PART B %%%%%%%%
    \notes\Ibl0553\qb05\hsk\na4\qb04\tbl0\qb05\Tqbu210\en\xbar%
    \notes\Ibu0163\qb01\hsk\na4\qb04\tbu0\qb06\Tqbl89{10}\en\xbar%
    \notes\Tqbl975\Ibl0553\qb05\hsk\na4\qb04\tbl0\qb05\en\xbar%
    \notes\Ibl0623\qb06\hsk\na4\qb04\tbl0\qb02\qup2\hsk\en\alaligne%
    %
    \notes\Ibu0533\qb05\hsk\na4\qb04\tbu0\qb03\Tqbu210\en\xbar%
    \notes\Ibu0d43\qb0d\qb01\hsk\na4\tbu0\qb04\Tqbl68{10}\en\xbar%
    \notes\Tqbl975\Ibl0843\qb08\qb06\hsk\na4\tbl0\qb04\en\xbar%
    \notes\qlp5\qup{c}\en\xbar%
    \notes\qlp5\Tqbl358\en\xbar%
    \notes\qlp6\Tqbl468\en\alaligne%
    %
    \notes\Tqbl578\Tqbl9{10}{11}\en\xbar%
    \notes\Tqbl{10}98\Tqbl765\en\xbar%
    \notes\qlp5\Tqbl358\en\xbar%
    \notes\qlp6\Tqbl468\en\xbar%
    \notes\Tqbl5{11}9\Tqbl757\en\xbar%
    \notes\qlp8\qup1\en\leftrightrepeat%
    \endpiece%
\end{music}

Ca{\s}t off \& hands 4 round with the 3.$^d$ Cu. \Thaaa ca{\s}t up \& hands 4 round
at top \Thbaa figure down contrary {\s}ides then on your own {\s}ides \Thaab lead down
2 Cu. \& ca{\s}t up lead thro' the top \& ca{\s}t up \Thbab
\HRule
\HRule
\vspace{1em}
\begin{description}
    \item[A1]
    \item[1-2]Popis v českém jazyce píšeme na doby odpovídající hudbě, kterou máme k dispozici
    \item[3-4]Používáme ustálená spojení u typických figur, kde to dává smysl
    \item[1-4]Jinak používáme natolik přesný popis, abychom ho sami pochopili kdybychom tanec neznali
    \item[5-8]Set and turn vlevo
    \item[9-16]Ditto, set and turn vpravo
    \item[B1]
    \item[1-2] Zapisujeme tanec v pořadí, v jakém má být tančen, nepřeskakujeme části
    \item[3-4] 1.P přejde k 2.D a podá levou ruku
    \item[5-6] 1.P přejde k 3.D a podá pravou, pak levou ruku
    \item[7-8] 1.P políbí 3.D na pravou a levou tvář
    \item[9-10] 1.P a 3.D zátočka za obě o celé, 2.P a 3.P step up
    \item[11-20] dtto vše, ale provádí 1.D, na konci 2.D a 3.D step up
    \item[A2]
    \item[1-16] Sides R, set and turn, sides L, set and turn
    \item[B2] = B1
    \item[A3]
    \item[1-16] Arms R, set and turn, arms L, set and turn
    \item[B3] = B1
\end{description}
\todo{Popis tance}