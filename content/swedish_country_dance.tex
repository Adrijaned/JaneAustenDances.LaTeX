\danceName{Swedish Country Dance}{long3s}
{\large\hfill Companion to La Terpsichore Moderne, J. S. Pollock, 1830}
\begin{center}
{\large Hudba k tomuto tanci není autorem specifikována}
\end{center}
\vspace{1em}
\HRule
\vspace{0.5em}
\begin{multicols}{2}
\begin{center}
\begin{tabular}{ c c c }
 O & X & O \\ 
 O & X & O \\ 
 O & X & O \\ 
 O & X & O \\ 
 O & X & O
\end{tabular}
\end{center}
This dance will be found particularly useful, where there happens to be a large majority of either ladies or gentlemen. The party being placed as above, in lines of three, a gent, and two ladies, or a lady and two gents, the dance proceeds in the same way as the Mescolanzes, except that no change of places is to be made either at the top or bottom of the same,

\columnbreak

\begin{description}
    \item[NO. 1. (4 parts)] Hands six round and back again -- the two top ladies and opposite gent. hands across and back again -- the other three the same -- advance and retire all six, and each cross over to the opposite person's place, which brings the first line into the place of the second, ready to repeat the same figure with the thirs line.
    \item[NO. 2. (4 parts)] All six advance and retire -- the two top ladies and opposite gent. hands three round and back again to places -- the other three the same -- advance and retire all six, and each cross to the opposite person's place.
    \item[NO. 3. (4 parts)] Hands six round and back again -- the top gent. swing the lady opposite on the right with his right hand, then the lady opposite on his left with his left hand -- the other gent. performs the same figure with the top ladies -- advance retire all six, and each cross the opposite person's place.
\end{description}
\end{multicols}
\vspace{1.5em}
\HRule
\HRule

\vspace{1.5em}
\begin{description}
    \item[1-8] Six hands circle doleva a doprava
    \item[9-16] 1. P zátočka za pravou s pravou 2. D, potom za levou s levou 2. D
    \item[17-24] Ditto 2. P s pravou a levou 1. D
    \item [25-32] 1x Advance and retire, výměna pravým ramenem
\end{description}