\danceNameS{Mi{\longs}s Molly's Wigg}{Miss Molly's Wigg}{longDMin}{thompson1774}%danceName must be followed by blank line for proper command parsing

{\large Longways double minor; $AABB$ adlib\hfill \Source{Thompson1774}}

\begin{music}
    \smallmusicsize%
    \parindent0mm%
    \raggedbottom%
    \stafftopmarg=1\Interligne%
    \staffbotmarg=1\Interligne%
    \nobarnumbers%
    \generalmeter{\meterfrac24}%
    \generalsignature{1}%
    \startpiece%
%%%%%%%% PART A %%%%%%%%
    \notes\Ibl0633\qbp06\tbbl0\qb04\qbp05\tbbl0\tbl0\qb03\en\xbar%
    \notes\slur42d1\Ibu0421\qbp04\tbbu0\tbu0\qb02\qu2\en\xbar%
    \notes\Ibl0633\qbp06\tbbl0\qb04\qbp05\tbbl0\tbl0\qb03\en\xbar%
    \notes\hl4\en\xbar%
    \notes\slur{d}5d3\Ibu0d53\qbp0d\tbbu0\qb01\qbp03\tbbu0\tbu0\qb05\en\xbar%
    %
    \notes\Ibu0421\qbp04\tbbu0\tbu0\qb02\tr9\ql7\en\xbar%
    \notes\Ibu0353\qbp03\tbbu0\qb01\qbp00\hsk\sh5\tbbu0\tbu0\qb05\en\xbar%
    \notes\hl6\en\leftrightrepeat\alaligne%
%%%%%%%% PART B %%%%%%%%
    \notes\Ibl0663\qb06\na8\qb08\qb07\tbl0\qb06\en\xbar%
    \notes\Dqbl75\ql5\en\xbar%
    %
    \notes\Qqbl7987\en\xbar%
    \notes\Ibl0861\qbp08\tbbl0\tbl0\qb06\ql6\en\xbar%
    \notes\Ibl0751\qbp07\tbbl0\tbl0\qb05\ql{10}\en\xbar%
    \notes\Ibl0641\qbp06\tbbl0\tbl0\qb04\ql9\en\xbar%
    \notes\Ibl0733\qbp07\tbbl0\qb05\qbp04\tbbl0\tbl0\qb03\en\xbar%
    \notes\hu2\en\rightrepeat%
    \endpiece%
\end{music}

Hands a cro{\longs}s foot it \Thaaa
hands a cro{\longs}s back again and foot it \Thbaa
cro{\longs}s over and half figure right Hands and left\Thaab\Thbab
\HRule
\HRule
\vspace{1em}
\begin{czechDescriptionTemplate}
\begin{description}
    \item[A1]
    \item[1-2]Popis v českém jazyce píšeme na doby odpovídající hudbě, kterou máme k dispozici
    \item[3-4]Používáme ustálená spojení u typických figur, kde to dává smysl
    \item[1-4]Jinak používáme natolik přesný popis, abychom ho sami pochopili kdybychom tanec neznali
    \item[5-8]Set and turn vlevo
    \item[9-16]Ditto, set and turn vpravo
    \item[B1]
    \item[1-2] Zapisujeme tanec v pořadí, v jakém má být tančen, nepřeskakujeme části
    \item[3-4] 1.P přejde k 2.D a podá levou ruku
    \item[5-6] 1.P přejde k 3.D a podá pravou, pak levou ruku
    \item[7-8] 1.P políbí 3.D na pravou a levou tvář
    \item[9-10] 1.P a 3.D zátočka za obě o celé, 2.P a 3.P step up
    \item[11-20] dtto vše, ale provádí 1.D, na konci 2.D a 3.D step up
    \item[A2]
    \item[1-16] Sides R, set and turn, sides L, set and turn
    \item[B2] = B1
    \item[A3]
    \item[1-16] Arms R, set and turn, arms L, set and turn
    \item[B3] = B1
\end{description}
\todo{Popis tance}
\end{czechDescriptionTemplate}