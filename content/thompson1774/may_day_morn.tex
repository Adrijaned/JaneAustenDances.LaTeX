\danceName{May Day Morn}{longDMin}{thompson1774}%danceName must be followed by blank line for proper command parsing

{\large Longways double minor; $AABBCC$ adlib\hfill \Source{Thompson1774}}

\begin{music}
    \smallmusicsize%
    \parindent0mm%
    \raggedbottom%
    \stafftopmarg=1\Interligne%
    \staffbotmarg=1\Interligne%
    \nobarnumbers%
    \generalmeter{\meterfrac68}%
    \generalsignature{1}%
    \startpiece%
%%%%%%%% PART A %%%%%%%%
    \notes\Tqbl642\ql9\cl6\en\xbar%
    \notes\Tqbu424\qu3\cl5\en\xbar%
    \notes\Tqbu426\ql9\cl5\en\xbar%
    \notes\Ibl06{10}2\qb04\qb06\tbu0\qb0d\Ibu0222\qb02\tbbu0\qb02\tbu0\qbp02\en\leftrightrepeat\alaligne%
%%%%%%%% PART B %%%%%%%%
    \notes\Tqbl868\tr{12}\ql{10}\cl9\en\xbar%
    \notes\Tqbl863\Ibl0732\qb07\hsk\sh5\qb05\tbl0\qb03\en\xbar%
    \notes\Tqbl868\ql9\cl5\en\xbar%
    \notes\Tqbu453\Ibu0222\qb02\tbbu0\qb02\tbu0\qbp02\en\leftrightrepeat\alaligne%
%%%%%%%% PART C %%%%%%%%
    \notes\qu4\cu{d}\ql5\cu{d}\en\xbar%
    \notes\Tqbu{d}13\Tqbl543\en\xbar%
    \notes\ql4\cu{d}\ql6\cl9\en\xbar%
    \notes\Ibl06{10}2\qb04\qb06\tbu0\qb0d\Ibu0222\qb02\tbbu0\qb02\tbu0\qbp02\en\leftrightrepeat%
    \endpiece%
\end{music}

The fir{\s}t man {\s}ets and turns the 2$.^d$ Wo: \Thaaa
the fir{\s}t Wo: {\s}et and turns the 2$.^d$ Man \Thbaa
cro{\s}s over one Cu: and lead up ca{\s}t back and turn \Thaab\Thbab
{\s}et and back to back foot it right hands and Left \Thaac\Thbac
\HRule
\HRule
\vspace{1em}
\begin{czechDescriptionTemplate}
\begin{description}
    \item[A1]
    \item[1-2]Popis v českém jazyce píšeme na doby odpovídající hudbě, kterou máme k dispozici
    \item[3-4]Používáme ustálená spojení u typických figur, kde to dává smysl
    \item[1-4]Jinak používáme natolik přesný popis, abychom ho sami pochopili kdybychom tanec neznali
    \item[5-8]Set and turn vlevo
    \item[9-16]Ditto, set and turn vpravo
    \item[B1]
    \item[1-2] Zapisujeme tanec v pořadí, v jakém má být tančen, nepřeskakujeme části
    \item[3-4] 1.P přejde k 2.D a podá levou ruku
    \item[5-6] 1.P přejde k 3.D a podá pravou, pak levou ruku
    \item[7-8] 1.P políbí 3.D na pravou a levou tvář
    \item[9-10] 1.P a 3.D zátočka za obě o celé, 2.P a 3.P step up
    \item[11-20] dtto vše, ale provádí 1.D, na konci 2.D a 3.D step up
    \item[A2]
    \item[1-16] Sides R, set and turn, sides L, set and turn
    \item[B2] = B1
    \item[A3]
    \item[1-16] Arms R, set and turn, arms L, set and turn
    \item[B3] = B1
\end{description}
\todo{Popis tance}
\end{czechDescriptionTemplate}