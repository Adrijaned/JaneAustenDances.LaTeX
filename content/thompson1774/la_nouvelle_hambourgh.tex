\danceName{La Nouvelle Hambourgh}{longTMin}{thompson1774}%danceName must be followed by blank line for proper command parsing

{\large Longways triple minor; $AABBCCDD$ adlib\hfill \Source{Thompson1774}}

\begin{music}
    \smallmusicsize%
    \parindent0mm%
    \raggedbottom%
    \stafftopmarg=1\Interligne%
    \staffbotmarg=1\Interligne%
    \nobarnumbers%
    \generalmeter{\meterfrac24}%
    \generalsignature{2}%
    \startpiece%
%%%%%%%% PART A %%%%%%%%
    \zleftrepeat%
    \notes\Qqbu{d}136\en\xbar%
    \notes\Qqbl8{10}97\en\xbar%
    \notes\Dqbl53\Qqbbl6567\en\xbar%
    \notes\ql6\ql6\en\xbar%
    \notes\Qqbu3210\en\xbar%
    \notes\Qqbu{d}136\en\xbar%
    \notes\Qqbl8{10}97\en\xbar%
    \notes\ql6\ql6\en\zzleftrightrepeat%
%%%%%%%% PART B %%%%%%%%
    \notes\Qqbl7536\en\xbar%
    \notes\Ibl0653\qb08\qb06\qb03\tbl0\qb08\en\xbar%
    \notes\Qqbl9876\en\xbar%
    \notes\Ibl0552\qb05\nbbl0\qb04\tbl0\qb05\qu3\en\xbar%
    \notes\Qqbl7536\en\xbar%
    \notes\Ibl0653\qb08\qb06\qb03\tbl0\qb08\en\xbar%
    \notes\Ibl0862\qb08\nbbl0\qb07\tbl0\qb06\Ibl0862\qb08\nbbl0\qb07\tbl0\qb06\en\xbar%
    \notes\ql6\ql6\en\zzleftrightrepeat%
%%%%%%%% PART C %%%%%%%%
    \notes\Qqbu10{d}0\en\xbar%
    \notes\Qqbu1236\en\xbar%
    \notes\Qqbu3210\en\xbar%
    \notes\Ibu01c3\qb01\qb00\qb0d\sh{c}\tbu0\qb0c\en\xbar%
    \notes\Qqbu10{d}0\en\xbar%
    \notes\Qqbu1236\en\xbar%
    \notes\sh5\Ibl0553\qb05\qb06\qb07\sh5\tbl0\qb05\en\xbar%
    \notes\ql6\ql6\en\zzleftrightrepeat%
%%%%%%%% PART D %%%%%%%%
    \notes\Qqbl8768\en\xbar%
    \notes\Qqbl9879\en\xbar%
    \notes\Ibl0863\qb08\qb06\sh5\qb05\tbl0\qb06\en\xbar%
    \notes\Ibl0751\qb07\sh5\tbl0\qb05\qu3\en\xbar%
    \notes\Qqbu10{d}0\en\xbar%
    \notes\Qqbu1236\en\xbar%
    \notes\sh5\Qqbl5679\en\xbar%
    \notes\Ibl0751\qb07\sh5\tbl0\qb05\ql6\en\leftrightrepeat%
    \endpiece%
\end{music}

Ca{\longs}t off 1 Cu. \& turn \Thaaa
lead thro' the bottom \& ca{\longs}t up \Thbaa
hands 6 round \Thaab
lead thro' \ye top \& ca{\longs}t off \Thbab
{\longs}et corners \& turn \Thaac\Thbac
lead out {\longs}ides\Thaad\Thbad
\HRule
\HRule
\vspace{1em}
\begin{czechDescriptionTemplate}
\begin{description}
    \item[A1]
    \item[1-2]Popis v českém jazyce píšeme na doby odpovídající hudbě, kterou máme k dispozici
    \item[3-4]Používáme ustálená spojení u typických figur, kde to dává smysl
    \item[1-4]Jinak používáme natolik přesný popis, abychom ho sami pochopili kdybychom tanec neznali
    \item[5-8]Set and turn vlevo
    \item[9-16]Ditto, set and turn vpravo
    \item[B1]
    \item[1-2] Zapisujeme tanec v pořadí, v jakém má být tančen, nepřeskakujeme části
    \item[3-4] 1.P přejde k 2.D a podá levou ruku
    \item[5-6] 1.P přejde k 3.D a podá pravou, pak levou ruku
    \item[7-8] 1.P políbí 3.D na pravou a levou tvář
    \item[9-10] 1.P a 3.D zátočka za obě o celé, 2.P a 3.P step up
    \item[11-20] dtto vše, ale provádí 1.D, na konci 2.D a 3.D step up
    \item[A2]
    \item[1-16] Sides R, set and turn, sides L, set and turn
    \item[B2] = B1
    \item[A3]
    \item[1-16] Arms R, set and turn, arms L, set and turn
    \item[B3] = B1
\end{description}
\todo{Popis tance}
\end{czechDescriptionTemplate}