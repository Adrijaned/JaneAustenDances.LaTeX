\danceNameS{The Fri{\s}ky}{The Frisky}{longTMin}{thompson1774}%danceName must be followed by blank line for proper command parsing

{\large Longways triple minor; $AABBCCDD$ adlib\hfill \Source{Thompson1774}}

\begin{music}
    \smallmusicsize%
    \parindent0mm%
    \raggedbottom%
    \stafftopmarg=1\Interligne%
    \staffbotmarg=1\Interligne%
    \nobarnumbers%
    \generalmeter{\meterfrac68}%
    \generalsignature{3}%
    \startpiece%
    % custom song macro
    \def\fN#1#2#3{\Ibl0#1#32\qbp0#1\tbbl0\qb0#2\tbl0\qb0#3}%
%%%%%%%% PART A %%%%%%%%
    \zleftrepeat%
    \notes\fN567\fN765\en\xbar%
    \notes\Tqbl456\ql5\cu3\en\xbar%
    \notes\fN567\fN765\en\xbar%
    \notes\Ibu0422\qbp04\tbbu0\qb03\tbu0\qb02\qup3\hsk\en\zzleftrightrepeat%
%%%%%%%% PART B %%%%%%%%
    \notes\Tqbl545\Tqbl545\en\xbar%
    \notes\Tqbl676\ql5\cu3\en\xbar%
    \notes\Tqbl545\Tqbl545\en\xbar%
    \notes\Ibu0422\qbp04\tbbu0\qb03\tbu0\qb02\qup3\hsk\en\zzleftrightrepeat%
%%%%%%%% PART C %%%%%%%%
    \notes\fN{o}po\fN9o9\en\xbar%
    \notes\fN898\fN876\en\xbar%
    \notes\fN567\fN876\en\xbar%
    \notes\fN564\qup3\hsk\en\zzleftrightrepeat%
%%%%%%%% PART D %%%%%%%%
    \notes\Ibu0002\qb00\zq{c}\qb03\tbu0\qb00\Ibu0112\qb01\zq{d}\qb03\tbu0\qb01\en\xbar%
    \notes\Tqbl456\Tqbl543\en\xbar%
    \notes\Ibu0002\qb00\zq{c}\qb03\tbu0\qb00\Ibu0112\qb01\zq{d}\qb03\tbu0\qb01\en\xbar%
    \notes\Tqbl654\qup3\hsk\en\leftrightrepeat%
    \endpiece%
\end{music}

Turn right hands \& ca{\s}t off 1 Cu. \Thaaa\Thaaa turn Left \& ca{\s}t off below the 3.$^d$ Cu. \Thbaa\Thbab
hands round 6 \Thaac\Thbac lead up to the top foot it \& ca{\s}t off \Thaad\Thbad
\HRule
\HRule
\vspace{1em}
\begin{czechDescriptionTemplate}
\begin{description}
    \item[A1]
    \item[1-2]Popis v českém jazyce píšeme na doby odpovídající hudbě, kterou máme k dispozici
    \item[3-4]Používáme ustálená spojení u typických figur, kde to dává smysl
    \item[1-4]Jinak používáme natolik přesný popis, abychom ho sami pochopili kdybychom tanec neznali
    \item[5-8]Set and turn vlevo
    \item[9-16]Ditto, set and turn vpravo
    \item[B1]
    \item[1-2] Zapisujeme tanec v pořadí, v jakém má být tančen, nepřeskakujeme části
    \item[3-4] 1.P přejde k 2.D a podá levou ruku
    \item[5-6] 1.P přejde k 3.D a podá pravou, pak levou ruku
    \item[7-8] 1.P políbí 3.D na pravou a levou tvář
    \item[9-10] 1.P a 3.D zátočka za obě o celé, 2.P a 3.P step up
    \item[11-20] dtto vše, ale provádí 1.D, na konci 2.D a 3.D step up
    \item[A2]
    \item[1-16] Sides R, set and turn, sides L, set and turn
    \item[B2] = B1
    \item[A3]
    \item[1-16] Arms R, set and turn, arms L, set and turn
    \item[B3] = B1
\end{description}
\todo{Popis tance}
\end{czechDescriptionTemplate}