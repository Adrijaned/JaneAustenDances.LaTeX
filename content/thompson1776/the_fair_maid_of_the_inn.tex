\danceNameS{The fair Maid of the Inn}{Fair Maid of the Inn, The}{longTMin}{thompson1776}%danceName must be followed by blank line for proper command parsing

{\large Longways triple minor; $AABB$ adlib\hfill\Source{Thompson1776}}

\begin{music}
    \smallmusicsize%
    \parindent0mm%
    \raggedbottom%
    \stafftopmarg=1\Interligne%
    \staffbotmarg=1\Interligne%
    \nobarnumbers%
    \generalmeter{\allabreve}%
    \generalsignature{-2}%
    \startpiece%
%%%%%%%% PART A %%%%%%%%
    \notes\slur46u1\Ibl0463\qb04\qb06\slur86u1\qb08\tbl0\qb06\ql4\ql4\en\xbar%
    \notes\slur56u1\Ibl0553\qb05\qb06\slur56u1\qb04\tbl0\qb06\qu3\qu1\en\xbar%
    \notes\slur24u1\Ibl0253\qb02\qb04\slur35u1\qb03\tbl0\qb05\slur46u1\Ibl0473\qb04\qb06\slur57u1\qb05\tbl0\qb07\en\xbar%
    \notes\Qqbl6876\tinynotesize\qu6\normalnotesize\hl5\en\alaligne%
    %
    \notes\slur46u1\Ibl0463\qb04\qb06\slur86u1\qb08\tbl0\qb06\ql4\ql4\en\xbar%
    \notes\slur56u1\Ibl0553\qb05\qb06\slur56u1\qb04\tbl0\qb06\slur43d1\tinynotesize\cu4\normalnotesize\qu3\slur21d1\Dqbu21\en\xbar%
    \notes\slur24u1\Ibl0253\qb02\qb04\slur35u1\qb03\tbl0\qb05\slur46u1\Ibl0453\qb04\qb06\slur65u1\qb06\tbl0\qb05\en\xbar%
    \notes\ql6\tr9\ql5\hl4\en\leftrightrepeat\alaligne%
%%%%%%%% PART B %%%%%%%%
    \notes\slur{d}0d1\Ibu0d43\qb0d\qb00\slur14d1\qb01\tbu0\qb04\qu2\qu2\en\xbar%
    \notes\slur12d1\Ibu0143\qb01\qb02\slur34d1\qb03\tbu0\qb04\ql5\ql5\en\xbar%
    \notes\slur45u1\Ibl0473\qb04\qb05\slur67u1\qb06\tbl0\qb07\slur8{11}u1\Ibl0983\qb08\qb0{11}\slur{10}9u1\qb0{10}\tbl0\qb09\en\xbar%
    \notes\slur87u1\Ibl0853\qb08\qb07\slur65u1\qb06\tbl0\qb05\ql4\ql4\en\alaligne%
    %
    \notes\slur{d}0d1\Ibu0d43\qb0d\qb00\slur14d1\qb01\tbu0\qb04\qu2\qu2\en\xbar%
    \notes\slur12d1\Ibu0143\qb01\qb02\slur34d1\qb03\tbu0\qb04\ql5\ql5\en\xbar%
    \notes\slur45u1\Ibl0473\qb04\qb05\slur67u1\qb06\tbl0\qb07\slur8{11}u1\Ibl0983\qb08\qb0{11}\slur{10}9u1\qb0{10}\tbl0\qb09\en\xbar%
    \notes\slur87u1\Ibl0853\qb08\qb07\slur65u1\qb06\tbl0\qb05\hl4\en\leftrightrepeat%
    \endpiece%
\end{music}

Ca{\s}t off 2 Cu. \& turn both hands \Thaa the {\s}ame back
again \Thba lead down up again \& ca{\s}t off \Thab lead thro'
the 3.$^d$ Cu. \& ca{\s}t up lead thro' the 2.$^d$ Cu. \& ca{\s}t off \Thbb
\HRule
\HRule
\vspace{1em}
\begin{czechDescriptionTemplate}
\begin{description}
    \item[A1]
    \item[1-2]Popis v českém jazyce píšeme na doby odpovídající hudbě, kterou máme k dispozici
    \item[3-4]Používáme ustálená spojení u typických figur, kde to dává smysl
    \item[1-4]Jinak používáme natolik přesný popis, abychom ho sami pochopili kdybychom tanec neznali
    \item[5-8]Set and turn vlevo
    \item[9-16]Ditto, set and turn vpravo
    \item[B1]
    \item[1-2] Zapisujeme tanec v pořadí, v jakém má být tančen, nepřeskakujeme části
    \item[3-4] 1.P přejde k 2.D a podá levou ruku
    \item[5-6] 1.P přejde k 3.D a podá pravou, pak levou ruku
    \item[7-8] 1.P políbí 3.D na pravou a levou tvář
    \item[9-10] 1.P a 3.D zátočka za obě o celé, 2.P a 3.P step up
    \item[11-20] dtto vše, ale provádí 1.D, na konci 2.D a 3.D step up
    \item[A2]
    \item[1-16] Sides R, set and turn, sides L, set and turn
    \item[B2] = B1
    \item[A3]
    \item[1-16] Arms R, set and turn, arms L, set and turn
    \item[B3] = B1
\end{description}
\todo{Popis tance}
\end{czechDescriptionTemplate}