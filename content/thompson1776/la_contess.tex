\danceNameS{La Conte{\s}s}{La Contess}{longTMin}{thompson1776}%danceName must be followed by blank line for proper command parsing

{\large Longways triple minor; $AABBCC$ adlib\hfill\Source{Thompson1776}}

\begin{music}
    \smallmusicsize%
    \parindent0mm%
    \raggedbottom%
    \stafftopmarg=1\Interligne%
    \staffbotmarg=1\Interligne%
    \nobarnumbers%
    \generalmeter{\meterfrac24}%
    \generalsignature{-2}%
    \startpiece%
%%%%%%%% PART A %%%%%%%%
    \notes\ql8\Qqbbl8987\en\xbar%
    \notes\ql6\Qqbbl6765\en\xbar%
    \notes\Qqbu4141\en\xbar%
    \notes\Qqbl468{11}\en\xbar%
    \notes\ql8\Qqbbl8987\en\xbar%
    \notes\ql6\Qqbbl6765\en\alaligne%
    %
    \notes\Dqbu41\Qqbbu2435\en\xbar%
    \notes\hl4\en\leftrightrepeat%
%%%%%%%% PART B %%%%%%%%
    \notes\Ibu0152\qb01\nbbu0\qb03\tbu0\qb05\Ibl0732\qb07\nbbl0\qb05\tbl0\qb03\en\xbar%
    \notes\Ibl0482\qb04\nbbl0\qb06\tbl0\qb08\Ibl0{11}62\qb0{11}\nbbl0\qb08\tbl0\qb06\en\xbar%
    \notes\Qqbl9753\en\xbar%
    \notes\Qqbbl4345\Dqbl44\en\alaligne%
    %
    \notes\Ibu0152\qb01\nbbu0\qb03\tbu0\qb05\Ibl0732\qb07\nbbl0\qb05\tbl0\qb03\en\xbar%
    \notes\Ibl0482\qb04\nbbl0\qb06\tbl0\qb08\Ibl0{11}62\qb0{11}\nbbl0\qb08\tbl0\qb06\en\xbar%
    \notes\Qqbl9753\en\xbar%
    \notes\hl4\en\leftrightrepeat%
%%%%%%%% PART C %%%%%%%%
    \notes\ql{11}\qu4\en\xbar%
    \notes\Dqbl88\qu1\en\alaligne%
    %
    \notes\Qqbl2435\en\xbar%
    \notes\Qqbbl4345\Dqbl44\en\xbar%
    \notes\Ibl0{11}62\qb0{11}\nbbl0\qb08\tbl0\qb06\Ibl0472\qb04\nbbl0\qb06\tbl0\qb07\en\xbar%
    \notes\Ibl0842\qb08\nbbl0\qb06\tbl0\qb04\Dqbu11\en\xbar%
    \notes\Qqbu2435\en\xbar%
    \notes\hu4\en\leftrightrepeat%
    \endpiece%
\end{music}

Ca{\s}t off 1 Cu. \& turn \Thaa lead thro' the bottom \&
ca{\s}t up \Thba hands 6 round \Thab lead thro' the top \&
ca{\s}t off \Thbb lead out {\s}ides \Thac\Thbc
\HRule
\HRule
\vspace{1em}
\begin{czechDescriptionTemplate}
\begin{description}
    \item[A1]
    \item[1-2]Popis v českém jazyce píšeme na doby odpovídající hudbě, kterou máme k dispozici
    \item[3-4]Používáme ustálená spojení u typických figur, kde to dává smysl
    \item[1-4]Jinak používáme natolik přesný popis, abychom ho sami pochopili kdybychom tanec neznali
    \item[5-8]Set and turn vlevo
    \item[9-16]Ditto, set and turn vpravo
    \item[B1]
    \item[1-2] Zapisujeme tanec v pořadí, v jakém má být tančen, nepřeskakujeme části
    \item[3-4] 1.P přejde k 2.D a podá levou ruku
    \item[5-6] 1.P přejde k 3.D a podá pravou, pak levou ruku
    \item[7-8] 1.P políbí 3.D na pravou a levou tvář
    \item[9-10] 1.P a 3.D zátočka za obě o celé, 2.P a 3.P step up
    \item[11-20] dtto vše, ale provádí 1.D, na konci 2.D a 3.D step up
    \item[A2]
    \item[1-16] Sides R, set and turn, sides L, set and turn
    \item[B2] = B1
    \item[A3]
    \item[1-16] Arms R, set and turn, arms L, set and turn
    \item[B3] = B1
\end{description}
\todo{Popis tance}
\end{czechDescriptionTemplate}