\danceName{The Royal Dozen}{longTMin}{thompson1776}%danceName must be followed by blank line for proper command parsing

{\large Longways triple minor, $AABB$ adlib\hfill \Source{Thompson1776}}  % E.g. {\large Longway for six; 3x AABB\hfill Playford, 1651}

\begin{music}
    \smallmusicsize%
    \parindent0mm%
    \raggedbottom%
    \stafftopmarg=1\Interligne%
    \staffbotmarg=1\Interligne%
    \nobarnumbers%
    \generalmeter{\meterfrac24}%
    \generalsignature{2}%
    \startpiece%
%%%%%%%% PART A %%%%%%%%
    \notes\Ibl0892\qb08\nbbl0\qb08\tbl0\qb09\Dqbl86\en\xbar%
    \notes\Ibl0782\qb07\nbbl0\qb07\tbl0\qb08\Dqbl73\en\xbar%
    \notes\Ibl0462\qb04\nbbl0\qb04\tbl0\qb06\Ibl0572\qb05\nbbl0\qb05\tbl0\qb07\en\xbar%
    \notes\Ibl0652\qb06\nbbl0\qb06\tbl0\qb05\Dqbl67\en\xbar%
    \notes\Qqbbl8789\Dqbl86\en\xbar%
    \notes\Qqbbl7678\Dqbl73\en\alaligne%
    %
    \notes\Ibl0462\qb04\nbbl0\qb04\tbl0\qb06\Ibl0572\qb05\nbbl0\qb05\tbl0\qb07\en\xbar%
    \notes\hl6\en\leftrightrepeat%
%%%%%%%% PART B %%%%%%%%
    \notes\cl{10}\ql{10}\cl8\en\xbar%
    \notes\Ibl0752\qb09\nbbl0\qb03\tbl0\qb04\Dqbu33\en\xbar%
    \notes\cl9\ql9\cl7\en\xbar%
    \notes\Ibl0872\uppz8\qb08\nbbl0\slur67u1\qb06\tbl0\qb07\Ibl0692\uppz6\qb06\nbbl0\slur89u1\qb08\tbl0\qb09\en\alaligne%
    %
    \notes\Ibl0{10}{11}2\uppz{10}\qb0{10}\nbbl0\slur{10}{11}u1\qb0{10}\tbl0\qb0{11}\Ibl0{10}92\uppz{10}\qb0{10}\nbbl0\slur89u1\qb08\tbl0\qb09\en\xbar%
    \notes\Ibl0{10}{11}2\uppz{10}\qb0{10}\nbbl0\slur{10}{11}u1\qb0{10}\tbl0\qb0{11}\Ibl0{10}62\uppz{10}\qb0{10}\nbbl0\slur86u1\qb08\tbl0\qb06\en\xbar%
    \notes\cl9\ql9\cl8\en\xbar%
    \notes\ql8\ql7\en\xbar%
    \notes\Ibl0892\qb08\nbbl0\qb08\tbl0\qb09\Dqbl86\en\xbar%
    \notes\Ibl0782\qb07\nbbl0\qb07\tbl0\qb08\Dqbl73\en\alaligne%
    %
    \notes\Ibl0462\qb04\nbbl0\qb04\tbl0\qb06\Ibl0572\qb05\nbbl0\qb05\tbl0\qb07\en\xbar%
    \notes\Ibl0652\qb06\nbbl0\qb06\tbl0\qb05\Dqbl67\en\xbar%
    \notes\Qqbbl8789\Dqbl86\en\xbar%
    \notes\Qqbbl7678\Dqbl73\en\xbar%
    \notes\Ibl0462\qb04\nbbl0\qb04\tbl0\qb06\Ibl0572\qb05\nbbl0\qb05\tbl0\qb07\en\xbar%
    \notes\hl6\en\leftrightrepeat%
    \endpiece%
\end{music}

Foot it 4 \& change {\s}ides \Thaaa the {\s}ame back again \Thbaa
lead down the middle up again \& ca{\s}t off \Thaba hands 4 at bottom right \& left at top \Thbba
\HRule
\HRule
\vspace{1em}
\begin{czechDescriptionTemplate}
\begin{description}
    \item[A1]
    \item[1-2]Popis v českém jazyce píšeme na doby odpovídající hudbě, kterou máme k dispozici
    \item[3-4]Používáme ustálená spojení u typických figur, kde to dává smysl
    \item[1-4]Jinak používáme natolik přesný popis, abychom ho sami pochopili kdybychom tanec neznali
    \item[5-8]Set and turn vlevo
    \item[9-16]Ditto, set and turn vpravo
    \item[B1]
    \item[1-2] Zapisujeme tanec v pořadí, v jakém má být tančen, nepřeskakujeme části
    \item[3-4] 1.P přejde k 2.D a podá levou ruku
    \item[5-6] 1.P přejde k 3.D a podá pravou, pak levou ruku
    \item[7-8] 1.P políbí 3.D na pravou a levou tvář
    \item[9-10] 1.P a 3.D zátočka za obě o celé, 2.P a 3.P step up
    \item[11-20] dtto vše, ale provádí 1.D, na konci 2.D a 3.D step up
    \item[A2]
    \item[1-16] Sides R, set and turn, sides L, set and turn
    \item[B2] = B1
    \item[A3]
    \item[1-16] Arms R, set and turn, arms L, set and turn
    \item[B3] = B1
\end{description}
\todo{Popis tance}
\end{czechDescriptionTemplate}