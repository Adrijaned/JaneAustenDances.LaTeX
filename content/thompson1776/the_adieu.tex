\danceName{The Adieu}{longDMin}{thompson1776}%danceName must be followed by blank line for proper command parsing

{\large Longways double minor; $AABBCCDD$ adlib\hfill \Source{Thompson1776}}

\begin{music}
    \smallmusicsize%
    \parindent0mm%
    \raggedbottom%
    \stafftopmarg=1\Interligne%
    \staffbotmarg=1\Interligne%
    \nobarnumbers%
    \generalmeter{\meterfrac24}%
    \generalsignature{-2}%
    \startpiece%
%%%%%%%% PART A %%%%%%%%
    \notes\Ibl0451\qbp04\tbbl0\tbl0\qb05\Ibl0642\qb06\nbbl0\qb05\tbl0\qb04\en\xbar%
    \notes\Ibu0532\qb05\nbbu0\qb04\tbu0\qb03\Ibu0422\qb04\nbbu0\qb03\tbu0\qb02\en\xbar%
    \notes\Dqbu10\Ibu0db2\qb0d\nbbu0\qb0c\tbu0\qb0b\en\xbar%
    \notes\Qqbl7665\en\alaligne%
    % inserted score break to fit it on the page
    \notes\Ibl0451\qbp04\tbbl0\tbl0\qb05\Ibl0642\qb06\nbbl0\qb05\tbl0\qb04\en\xbar%
    \notes\Ibu0532\qb05\nbbu0\qb04\tbu0\qb03\Ibu0422\qb04\nbbu0\qb03\tbu0\qb02\en\xbar%
    \notes\Dqbu10\Ibbl0273\qb0d\qb09\qb08\tbl0\qb07\en\xbar% replaced direction-changing beam because MusixTex didn't like it
    \notes\Qqbbl6543\ql4\en\leftrightrepeat\alaligne%
%%%%%%%% PART B %%%%%%%%
    \notes\Qqbbl8767\Dqbl88\en\xbar%
    \notes\Qqbl8{11}{11}{10}\en\xbar%
    \notes\Qqbl{10}998\en\xbar%
    \notes\Dqbl87\ql6\en\alaligne%
    % inserted score break to fit it on the page
    \notes\Qqbbl8789\Ibl0882\qbp08\tbbl0\tbl0\qb08\en\xbar%
    \notes\Qqbl8{11}47\en\xbar%
    \notes\ql6\ql5\en\xbar%
    \notes\hl4\en\leftrightrepeat\alaligne%
%%%%%%%% PART C %%%%%%%%
    \notes\Ibu0bd3\qbp0b\tbbu0\qb0c\qb0d\tbu0\qb0d\en\xbar%
    \notes\Qqbu{d}eff\en\xbar%
    \notes\Ibl0443\qbp04\tbbl0\qb04\qbp06\tbbl0\tbl0\qb04\en\xbar%
    \notes\Dqbu31\qu1\en\alaligne%
    % inserted score break to fit it on the page
    \notes\Ibu0bd3\qbp0b\tbbu0\qb0c\qb0d\tbu0\qb0d\en\xbar%
    \notes\Dqbu{d}f\Qqbbu3210\en\xbar%
    \notes\qu{d}\qu{c}\en\xbar%
    \notes\hu{b}\en\leftrightrepeat\alaligne%
%%%%%%%% PART D %%%%%%%%
    \notes\Ibl0872\qb08\nbbl0\qb08\tbl0\qb07\Dqbl67\en\xbar%
    \notes\Ibl08{11}2\qbp08\tbbl0\qb0{11}\qbp08\tbbl0\tbl0\qb0{11}\en\xbar%
    \notes\Qqbbl{10}987\Ibl0671\qbp06\tbbl0\tbl0\qb07\en\xbar%
    \notes\Dqbl86\ql4\en\alaligne%
    % inserted score break to fit it on the page
    \notes\Ibl0872\qb08\nbbl0\qb08\tbl0\qb07\Dqbl67\en\xbar%
    \notes\Qqbbl89{10}{11}\Qqbbl{11}864\en\xbar%
    \notes\ql6\tr{10}\ql5\en\xbar%
    \notes\hl4\en\leftrightrepeat%
    \endpiece%
\end{music}

Foot it all 4 \& change places \Thaa the {\s}ame back again \Thba lead down the
middle \Thab up again \& ca{\s}t off \Thbb fall in bott: \& top \& foot it then
{\s}ide ways \Thac Allemand right hands then Allemand left hands \Thac%Yep, \Thac both symbols in original, probably typo
\HRule
\HRule
\vspace{1em}
\begin{czechDescriptionTemplate}
\begin{description}
    \item[A1]
    \item[1-2]Popis v českém jazyce píšeme na doby odpovídající hudbě, kterou máme k dispozici
    \item[3-4]Používáme ustálená spojení u typických figur, kde to dává smysl
    \item[1-4]Jinak používáme natolik přesný popis, abychom ho sami pochopili kdybychom tanec neznali
    \item[5-8]Set and turn vlevo
    \item[9-16]Ditto, set and turn vpravo
    \item[B1]
    \item[1-2] Zapisujeme tanec v pořadí, v jakém má být tančen, nepřeskakujeme části
    \item[3-4] 1.P přejde k 2.D a podá levou ruku
    \item[5-6] 1.P přejde k 3.D a podá pravou, pak levou ruku
    \item[7-8] 1.P políbí 3.D na pravou a levou tvář
    \item[9-10] 1.P a 3.D zátočka za obě o celé, 2.P a 3.P step up
    \item[11-20] dtto vše, ale provádí 1.D, na konci 2.D a 3.D step up
    \item[A2]
    \item[1-16] Sides R, set and turn, sides L, set and turn
    \item[B2] = B1
    \item[A3]
    \item[1-16] Arms R, set and turn, arms L, set and turn
    \item[B3] = B1
\end{description}
\todo{Popis tance}
\end{czechDescriptionTemplate}