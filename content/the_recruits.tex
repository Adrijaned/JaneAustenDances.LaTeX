\danceName{The Recruits}{longDMin}{thompson1774}%danceName must be followed by blank line for proper command parsing

{\large Longways double minor; $AABB$ adlib\hfill \Source{Thompson1774}}

\begin{music}
    \smallmusicsize%
    \parindent0mm%
    \raggedbottom%
    \stafftopmarg=1\Interligne%
    \staffbotmarg=1\Interligne%
    \nobarnumbers%
    \generalmeter{\meterfrac24}%
    \generalsignature{0}%
    \startpiece%
%%%%%%%% PART A %%%%%%%%
    \zleftrepeat%
    \notes\Qqbbl95{10}5\Qqbbl95{12}5\en\xbar%
    \notes\Qqbbl959{12}\Qqbbl{10}987\en\xbar%
    \notes\Qqbbl979{12}\Qqbbl95{12}5\en\xbar%
    \notes\Qqbbl7987\Dqbl66\en\zzleftrightrepeat%
%%%%%%%% PART B %%%%%%%%
    \notes\Qqbbu2053\Qqbbu2101\en\xbar%
    \notes\Qqbbu2053\Qqbbu2101\en\xbar%
    \notes\Qqbbu2053\Qqbbu2101\en\xbar%
    \notes\Qqbbl6876\ql5\en\leftrightrepeat%
    \endpiece%
\end{music}

Right hands acro{\s}s quite round \Thaaa left hands back again \Thbaa
cro{\s}s over 1 Cu. \& turn \Thaab Right \& Left \Thbab
\HRule
\HRule
\vspace{1em}
\begin{description}
    \item[A1]
    \item[1-2]Popis v českém jazyce píšeme na doby odpovídající hudbě, kterou máme k dispozici
    \item[3-4]Používáme ustálená spojení u typických figur, kde to dává smysl
    \item[1-4]Jinak používáme natolik přesný popis, abychom ho sami pochopili kdybychom tanec neznali
    \item[5-8]Set and turn vlevo
    \item[9-16]Ditto, set and turn vpravo
    \item[B1]
    \item[1-2] Zapisujeme tanec v pořadí, v jakém má být tančen, nepřeskakujeme části
    \item[3-4] 1.P přejde k 2.D a podá levou ruku
    \item[5-6] 1.P přejde k 3.D a podá pravou, pak levou ruku
    \item[7-8] 1.P políbí 3.D na pravou a levou tvář
    \item[9-10] 1.P a 3.D zátočka za obě o celé, 2.P a 3.P step up
    \item[11-20] dtto vše, ale provádí 1.D, na konci 2.D a 3.D step up
    \item[A2]
    \item[1-16] Sides R, set and turn, sides L, set and turn
    \item[B2] = B1
    \item[A3]
    \item[1-16] Arms R, set and turn, arms L, set and turn
    \item[B3] = B1
\end{description}
\todo{Popis tance}