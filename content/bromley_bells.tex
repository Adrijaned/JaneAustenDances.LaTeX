\danceName{Bromley Bells}{longDMin}{thompson1774}%danceName must be followed by blank line for proper command parsing

{\large Longways double minor; $AABB$ adlib\hfill \Source{Thompson1774}}

\midifyable
\begin{music}
    \smallmusicsize%
    \parindent0mm%
    \raggedbottom%
    \stafftopmarg=1\Interligne%
    \staffbotmarg=1\Interligne%
    \nobarnumbers%
    \generalmeter{\allabreve}%
    \generalsignature{-1}%
    \startpiece%
%%%%%%%% PART A %%%%%%%%
    \zleftrepeat%
    \notes\Dqbl35\ql5\Qqbl6424\en\xbar%
    \notes\Qqbl5{11}97\Ibbl0893\slur89u3\qb08\qb07\qb08\tbl0\qb09\Dqbl{10}8\en\xbar%
    \notes\Dqbl35\ql5\Qqbl6428\en\alaligne%
    %
    \notes\Ibl0743\qb07\qb05\qb06\na4\tbl0\qb04\ql5\qu{c}\en\leftrightrepeat%
%%%%%%%% PART B %%%%%%%%
    \notes\ql9\Dqbl75\ql{10}\Dqbl85\en\xbar%
    \notes\ibu01{-3}\qb04\qb02\tbl0\qb0{11}\tbl0\qb09\Ibl0853\qb08\qb07\qb06\sh5\tbl0\qb05\en\alaligne%
    %
    \notes\ibl060\qb06\qb03\qb08\tbl0\qb06\ibl080\qb09\qb07\qb0{11}\tbl0\qb09\en\xbar%
    \notes\ibl08{-3}\qb08\qb07\qb06\sh5\tbl0\qb05\ql6\ibu024\qb0d\nbbu0\qb03\tbu0\qb04\en\xbar%
    \notes\Ibl0583\qb05\qb03\qb0{10}\tbl0\qb08\Ibl0693\qb06\qb04\qb0{11}\tbl0\qb09\en\alaligne%
    %
    \notes\Qqbl{10}878\Ibbl0963\slur96u3\qb09\qb08\qb07\tbl0\qb06\ql5\en\xbar%
    \notes\ibl060\qb06\qb08\qb04\tbl0\qb06\Qqbl579{11}\en\xbar%
    \notes\Qqbl{10}857\ql8\qu1\en\leftrightrepeat%
    \endpiece%
\end{music}

Right hands acro{\s}s \Thaaa Left hands back again \Thbaa ca{\s}t off 1 Cu. \& turn 2.$^d$ Cu. do the
    {\s}ame \Thaab lead down 2 Cu. up 1 \& Right \& Left \Thbab
\HRule
\HRule
\vspace{1em}
\begin{description}
    \item[A1]
    \item[1-2]Popis v českém jazyce píšeme na doby odpovídající hudbě, kterou máme k dispozici
    \item[3-4]Používáme ustálená spojení u typických figur, kde to dává smysl
    \item[1-4]Jinak používáme natolik přesný popis, abychom ho sami pochopili kdybychom tanec neznali
    \item[5-8]Set and turn vlevo
    \item[9-16]Ditto, set and turn vpravo
    \item[B1]
    \item[1-2] Zapisujeme tanec v pořadí, v jakém má být tančen, nepřeskakujeme části
    \item[3-4] 1.P přejde k 2.D a podá levou ruku
    \item[5-6] 1.P přejde k 3.D a podá pravou, pak levou ruku
    \item[7-8] 1.P políbí 3.D na pravou a levou tvář
    \item[9-10] 1.P a 3.D zátočka za obě o celé, 2.P a 3.P step up
    \item[11-20] dtto vše, ale provádí 1.D, na konci 2.D a 3.D step up
    \item[A2]
    \item[1-16] Sides R, set and turn, sides L, set and turn
    \item[B2] = B1
    \item[A3]
    \item[1-16] Arms R, set and turn, arms L, set and turn
    \item[B3] = B1
\end{description}
\todo{Popis tance}